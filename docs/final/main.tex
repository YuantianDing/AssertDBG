%%%%%%%% ICML 2025 EXAMPLE LATEX SUBMISSION FILE %%%%%%%%%%%%%%%%%

\documentclass{article}

% Recommended, but optional, packages for figures and better typesetting:
\usepackage{microtype}
\usepackage{graphicx}
\usepackage{subfigure}
\usepackage{booktabs} % for professional tables

% hyperref makes hyperlinks in the resulting PDF.
% If your build breaks (sometimes temporarily if a hyperlink spans a page)
% please comment out the following usepackage line and replace
% \usepackage{icml2025} with \usepackage[nohyperref]{icml2025} above.
\usepackage{hyperref}

\usepackage{pythonhighlight}
% Attempt to make hyperref and algorithmic work together better:
\newcommand{\theHalgorithm}{\arabic{algorithm}}

% Use the following line for the initial blind version submitted for review:
\usepackage{icml2025}

% If accepted, instead use the following line for the camera-ready submission:
% \usepackage[accepted]{icml2025}

% For theorems and such
\usepackage{amsmath}
\usepackage{amssymb}
\usepackage{mathtools}
\usepackage{amsthm}

% if you use cleveref..
\usepackage[capitalize,noabbrev]{cleveref}

%%%%%%%%%%%%%%%%%%%%%%%%%%%%%%%%
% THEOREMS
%%%%%%%%%%%%%%%%%%%%%%%%%%%%%%%%
\theoremstyle{plain}
\newtheorem{theorem}{Theorem}[section]
\newtheorem{proposition}[theorem]{Proposition}
\newtheorem{lemma}[theorem]{Lemma}
\newtheorem{corollary}[theorem]{Corollary}
\theoremstyle{definition}
\newtheorem{definition}[theorem]{Definition}
\newtheorem{assumption}[theorem]{Assumption}
\theoremstyle{remark}
\newtheorem{remark}[theorem]{Remark}

% Todonotes is useful during development; simply uncomment the next line
%    and comment out the line below the next line to turn off comments
%\usepackage[disable,textsize=tiny]{todonotes}
\usepackage[textsize=tiny]{todonotes}

\icmltitlerunning{ Debugging like an Expert: Assertion-based Large Language Model Debugging}

\begin{document}

\twocolumn[
\icmltitle{Debugging like an Expert: \\ Assertion-based Large Language Model Debugging}

\icmlsetsymbol{equal}{*}

\begin{icmlauthorlist}
\icmlauthor{Firstname1 Lastname1}{equal,yyy}
\icmlauthor{Firstname2 Lastname2}{equal,yyy,comp}
\icmlauthor{Firstname3 Lastname3}{comp}
\icmlauthor{Firstname4 Lastname4}{sch}
\icmlauthor{Firstname5 Lastname5}{yyy}
\icmlauthor{Firstname6 Lastname6}{sch,yyy,comp}
\icmlauthor{Firstname7 Lastname7}{comp}
%\icmlauthor{}{sch}
\icmlauthor{Firstname8 Lastname8}{sch}
\icmlauthor{Firstname8 Lastname8}{yyy,comp}
%\icmlauthor{}{sch}
%\icmlauthor{}{sch}
\end{icmlauthorlist}

\icmlaffiliation{yyy}{Department of XXX, University of YYY, Location, Country}
\icmlaffiliation{comp}{Company Name, Location, Country}
\icmlaffiliation{sch}{School of ZZZ, Institute of WWW, Location, Country}

\icmlcorrespondingauthor{Anonymous Authors}{}
% \icmlcorrespondingauthor{Firstname2 Lastname2}{first2.last2@www.uk}

\icmlkeywords{Machine Learning, ICML}

\vskip 0.3in
]

% this must go after the closing bracket ] following \twocolumn[ ...

% This command actually creates the footnote in the first column
% listing the affiliations and the copyright notice.
% The command takes one argument, which is text to display at the start of the footnote.
% The \icmlEqualContribution command is standard text for equal contribution.
% Remove it (just {}) if you do not need this facility.

%\printAffiliationsAndNotice{}  % leave blank if no need to mention equal contribution
\printAffiliationsAndNotice{\icmlEqualContribution} % otherwise use the standard text.

\begin{abstract}
    Large Language Models (LLMs) have shown impressive capabilities in code generation, yet they frequently produce programs with subtle, hard-to-detect bugs. To address this challenge, I proposed an assertion-based debugging framework that emulates expert debugging strategies to identify and correct errors in LLM-generated code. Our approach leverages a multi-agent architecture: a divide-and-conquer agent decomposes complex problems into smaller, tractable subtasks; programmer agents generate code enriched with strategic assertions; and a debugger agent iteratively analyzes assertion failures to refine incorrect outputs. By incorporating assertions into the debugging process, our framework improves robustness and correctness, achieving a 3\% performance gain on the HumanEvalPlus benchmark suite.
\end{abstract}



\section{Introduction}

Code generation and program synthesis have emerged as transformative fields within machine learning, aiming to automate the process of software development. By leveraging advanced neural architectures—particularly large language models (LLMs)—these approaches translate natural language descriptions or formal specifications into executable code. This automation not only accelerates development cycles but also opens avenues for reducing human error and enhancing productivity in complex programming tasks.

Recent advancements in LLMs have catalyzed transformative progress in automated code generation. While early approaches primarily focused on one-shot generation, emerging methodologies increasingly emphasize execution-based checking and multi-agent collaboration to address challenges such as subtle semantic errors and logical inconsistencies. In this context, a growing body of research explores mechanisms for self-debugging, testing, and multi-agent coordination to enhance both the reliability and efficiency of program synthesis.

Notably, \cite{TeachSelfDebug} demonstrates that instructing LLMs to engage in self-reflection and error diagnosis can markedly improve the correctness of generated code. Complementing this, \cite{LEVER} introduces an execution-driven approach for assessing code accuracy, where the generation process is iteratively refined based on dynamic feedback obtained through code execution. This execution-based approach is further advanced in \cite{LDB}, which emulates human debugging by systematically verifying runtime behavior in a stepwise manner. 

Parallel to these self-debugging and testing strategies, multi-agent frameworks have emerged as a powerful paradigm for code generation. Systems such as \cite{MetaGPT, MapCoder, QualityFlow} proposes an multi-agent workflow that integrates software engineering methodologies into the program synthesis process, including role specialization, structured communication, and iterative refinement. Similar architectures are also proposed in \cite{AgentCoder} and \cite{CODESIM}. Collectively, these studies underscore a paradigm shift from static, single-shot code generation towards dynamic, collaboration-driven methodologies. By combining self-debugging techniques with execution-based feedback and multi-agent coordination, recent research provides a promising roadmap for developing more robust, reliable, and adaptive code generation systems, laying the groundwork for future advancements in autonomous programming.

Despite recent advances, several critical aspects of human debugging remain underexplored in current state-of-the-art methodologies. In practice, expert programmers routinely incorporate assertions into their code as a proactive measure to enforce correctness, thereby enabling the early detection and localization of bugs during runtime. These runtime checks not only streamline the debugging process but also enhance overall software reliability. For instance, languages like Java and Python automatically perform array boundary checks, offering a layer of protection that languages such as C do not inherently provide. In this project, we aim to investigate the efficacy of integrating assertion-guided strategies into LLM-driven code generation, leveraging these human-inspired debugging techniques to improve both the robustness and accuracy of synthesized code.

Here we list several research questions that we aim to address in this project:

\textbf{RQ1:} How can we effectively incorporate assertion-guided strategies into LLM-based code generation systems to enhance the reliability and correctness of synthesized code?

\textbf{RQ2:} To what extent do assertion-guided methodologies improve the efficiency and scalability of program synthesis, particularly in complex, large-scale applications?

\textbf{RQ3:} How can assertion-guided techniques be integrated with existing self-debugging, runtime checking, and multi-agent coordination strategies to develop a comprehensive, end-to-end framework for code generation that enhances both reliability and efficiency?






\section{Motivation}

\subsection{Problem Definition}

I follow the problem definition of code generation in \cite{TeachSelfDebug}. Basically, each sample
can be represented as a triplet $(Q, T_v, T_h)$ where $Q$ is the natural language description of the task, $T_v$ is the target code, and $T_h$ is the code generated by the model. $Q$ may include a description of the task, the signature of the function to be implemented, and the constraints that the code must satisfy. In standard code generation tasks, both $Q$ and $T_v$ are provided as inputs, and the objective is to generate a program $P$ that meets the specifications of both $T_v$ and $T_h$. The generated program $P$ is evaluated against both $T_v$ and $T_h$ to check its correctness.

In this project, I specifically aim to improve the correctness of code generated by large language models (LLMs). Achieving correctness is particularly challenging for LLMs because, despite their impressive fluency and syntactic capabilities, they often lack a deep semantic understanding of the tasks they are solving. As a result, LLMs can produce code that appears syntactically valid and may even pass a limited set of test cases, yet still fail to fully capture the problem's intent or handle critical edge cases.

To address this issue, I explore multi-agent collaboration-based testing methods, which enhance reliability by introducing diverse perspectives into the validation process. However, LLMs frequently struggle to generate robust and accurate test cases, primarily due to a lack of sufficient task-specific data or reasoning depth.

To illustrate this limitation, I randomly selected one benchmark problem \texttt{HumanEval/118} from the HumanEvalPlus benchmark suite. Using the \texttt{gpt-4o} model, I generated a corresponding test function. As shown in Figure~\ref{fig:program}, although the model is capable of generating a correct solution for this problem, it fails to produce valid test cases. 

In this project, I propose assertion-based debugging to address this challenge. By embedding runtime assertions throughout the code, I can expose hidden errors that might be overlooked by surface-level test cases. These assertions act as lightweight contracts that enforce input-output constraints, loop invariants, and other correctness properties during execution. When an assertion fails, it provides a precise signal to a debugger agent, which then locates and fixes the faulty function in an iterative loop. This structured approach mimics human debugging strategies and enhances both the reliability and interpretability of LLM-generated code.


\begin{figure*}[t]
\begin{python}
def get_closest_vowel(word):
    """You are given a word. Your task is to find the closest vowel that stands between 
    two consonants from the right side of the word (case sensitive).
    
    Vowels in the beginning and ending doesn't count. Return empty string if you didn't
    find any vowel met the above condition. 

    You may assume that the given string contains English letter only.

    Example:
    get_closest_vowel("yogurt") ==> "u"
    get_closest_vowel("FULL") ==> "U"
    get_closest_vowel("quick") ==> ""
    get_closest_vowel("ab") ==> ""
    """
    def is_vowel(ch: str) -> bool:
        return ch in "aeiouAEIOU"
    for i in range(len(word) - 2, 0, -1):
        if is_vowel(word[i]) and not is_vowel(word[i-1]) and not is_vowel(word[i+1]):
            return word[i]
    return ""

def test_get_closest_vowel():
    assert get_closest_vowel("yogurt") == "u"
    assert get_closest_vowel("FULL") == "U"
    assert get_closest_vowel("planet") == "e"
    assert get_closest_vowel("ab") == ""
    assert get_closest_vowel("HeLLo") == ""  # INCORRECT !!!
    assert get_closest_vowel("ABcdEF") == "" # INCORRECT !!!
    assert get_closest_vowel("a") == ""
    ......

\end{python}
\caption{Example of incorrect test cases generated by \texttt{gpt-4o} while the program can be correctly generated. }
\label{fig:program}
\end{figure*}


 


\subsection{Existing Approaches}


Program synthesis and code generation is certainly a wide-researched field, with a plethora of approaches and methodologies. In this section, we select several notable work both from top conferences and from state-of-the-art leaderboards \cite{PaperWithCode}.

\textsc{Self-Debugging} \cite{TeachSelfDebug} introduces an innovative approach to enhancing the accuracy of code generated by large language models (LLMs). The core idea involves enabling LLMs to autonomously execute their generated code using external programs, analyze the outcomes, identify errors, and subsequently refine their code based on this self-assessment. This self-debugging process allows the model to iteratively improve its code without external feedback, such as unit tests or human instructions. However, the self-debugging approach may failed to utilizing runtime execution information to enhance the accuracy and reliability of code generation, as mentioned by other works \cite{LDB}.

Large Language Model Debugger (LDB) \cite{LDB} is a framework designed to emulate human debugging practices by integrating runtime execution information into the debugging process. Unlike traditional methods that treat programs as monolithic units, LDB divides programs into smaller segments, allowing for more granular analysis and refinement. By adopting this step-by-step approach, LDB effectively mirrors human debugging strategies, focusing on smaller code units and utilizing runtime information to enhance the accuracy and reliability of code generation. However, LDB's fine-grained approach may introduce additional complexity and computational overhead, potentially limiting its scalability to large-scale applications. With this in mind, our assertion-based debugging framework aims to strike a balance between granularity and efficiency, leveraging assertions to guide the debugging process while maintaining scalability and robustness across diverse programming tasks.

Another notable work is MetaGPT \cite{MetaGPT} which leverages large language models (LLMs) to facilitate multi-agent collaboration in complex software engineering tasks. By integrating standardized human workflows into LLM-based multi-agent systems, MetaGPT assigns specialized roles—such as product manager, architect, and engineer—to different agents, mirroring traditional software development pipelines. This role specialization enables the decomposition of intricate tasks into manageable subtasks, promoting coherent and efficient problem-solving. The framework employs an assembly line paradigm, where agents with domain-specific expertise collaborate systematically, reducing errors and enhancing the overall quality of the generated solutions. However, MetaGPT does not explicitly address the issue of runtime error detection and prevention, focusing primarily on role specialization and structured communication. In contrast, our assertion-based debugging framework aims to proactively prevent errors during the synthesis process by incorporating assertion-guided strategies, complementing existing multi-agent methodologies.

QualityFlow \cite{QualityFlow} is a recent state-of-the-art work on \cite{PaperWithCode} that emphasizes the importance of quality assurance in code generation. By introducing a quality control mechanism that evaluates the correctness and robustness of generated code, QualityFlow ensures that the synthesized solutions meet predefined quality standards. The framework employs a feedback loop that iteratively refines the generation process based on quality metrics, enhancing the reliability and accuracy of the final output. While QualityFlow focuses on post-generation quality assessment, our assertion-based debugging framework aims to proactively prevent errors during the synthesis process by incorporating assertion-guided strategies. By integrating runtime checks and assertions into the code generation pipeline, we aim to enhance the reliability and correctness of synthesized code, complementing existing quality assurance methodologies.



\section{Methodology}

\subsection{Overall Architecture}

\begin{figure*}
    \centering
    \includegraphics[width=\linewidth]{Architecture.pdf}
    \caption{Overview of the proposed assertion-based debugging framework.}
    \label{fig:overview}
\end{figure*}

Figure~\ref{fig:overview} illustrates an assertion-based debugging framework for large language models (LLMs), which adopts a divide-and-conquer strategy for both problem-solving and debugging. The process begins with a divide-and-conquer agent that decomposes the overall problem specification into a set of smaller subtask specifications. Each subtask is represented as a pair $(Q_i, A_i)$, where $Q_i$ is a natural language description of the subtask, and $A_i$ is a set of assertions that any valid solution must satisfy.

In addition to generating these subtasks, the divide-and-conquer agent also produces the code required to assemble the individual subprograms into a unified solution. Each subtask is then handled by a programmer agent, which implements a corresponding function enriched with detailed runtime assertions. These assertions are designed to catch potential errors and enforce the correctness of the solution.

To verify the correctness of the assembled program, a tester agent generates a single unit test function. This function combines predefined and automatically generated test cases to evaluate the entire solution in a global testing phase. If all tests pass, the solution is accepted. Otherwise, if an assertion fails, a debugger agent is invoked.

The debugger agent is the core of this framework. It analyzes the error message to locate the source of the bug by navigating the codebase. Once sufficient information is gathered, it proposes a fix by replacing one or more faulty functions with corrected versions. The testing and debugging cycle continues iteratively until a correct solution is found. This framework mirrors expert-level debugging practices by embedding assertions, systematically isolating faults, and refining the solution step by step.

In some cases, the debugger agent may be unable to localize the issue based solely on the error message—typically due to a misunderstanding of the original problem specification. When this occurs, the entire process is restarted from scratch to produce a new candidate solution.

% It is worth noting that, the debugger agent, the tester agent and the programmer agents in the framework can work \textit{concurrently} without depending on the completion of each other. This parallel processing capability enhances the efficiency and scalability of the framework, enabling rapid debugging and solution generation for complex programming tasks. The framework's modular design allows for easy integration of additional agents or functionalities, making it adaptable to diverse programming scenarios.

\subsection{Prompt Design}

Since the proposed framework requires a large amount of assertions to generated by the programmer agent, the prompt used to guide the LLMs must be carefully designed to produces desired assertions. We must specify exactly what kind of assertions we want the LLMs to generate, and how to incorporate these assertions into the generated code. The prompt design should also consider the balance between the number of assertions and the complexity of the generated code, as an excessive number of assertions may lead to code bloat and performance degradation.

Here I list the important categories of assertions that I aim to specify in the prompt:

\begin{enumerate}
    \item \textbf{Input Assertions} specify the constraints that the input data must satisfy. These assertions are especially useful for checking the correctness of the combination program generated by the divide-and-conquer agent.
    \item \textbf{Output Assertions} specify the expected output of the program. A lot of engineering efforts need to put into those assertions, as they are the most important part of the program. Thus I force the model to generate such assertions, and recommand the model to create helper functions to check the output match the specialization exactly. For example, in a sorting algorithm, an output assertion may specify that the output list must be sorted in ascending order, which may involves a helper function \texttt{isSorted} to ensure that list is completely sorted.
    \item \textbf{Loop Invariants} can potentially help the model to understand the loop structure and the expected behavior of the loop without using information from long-distance context, and thus generate more accurate assertions. 
\end{enumerate}

Similarly, the prompting strategies used for both the tester agent and the debugger agent play a crucial role in the success of the framework. As illustrate in Figure~\ref{fig:program} one common issue arises when test cases generated by the tester agent are themselves flawed. To mitigate this, I instruct the tester agent to rely primarily on random testing, which helps reduce overfitting to a narrow interpretation of the problem. In this setup, the program specification is effectively written twice: once in the implementation prompt and again in the test generation prompt. This redundancy encourages coverage diversity, but also introduces the potential for misalignment between the two representations of the same task.

Given this setup, the debugger agent must be carefully guided to reason across multiple potential sources of error: the program implementation, the embedded assertions, and the test script itself. It must not only identify the component responsible for the failure but also avoid unnecessarily altering correct parts of the system. To support this, the debugger agent is carefully designed to analyze error messages and assertion failures in a structured manner, navigating the codebase using introspection tools to locate faults. By explicitly including the possibility of errors in any part of the development pipeline—including faulty tests or incorrect assertions - the framework ensures that debugging is not biased toward assuming code is always the source of failure.





\section{Experimental Result}

Now I present the results of our assertion-based debugging framework on the HumanEvalPlus benchmark. I use a simplified version of the architecture in Figure \ref{fig:overview} with OpenAI \texttt{gpt-4o}. The result is shown in Table \ref{tab:result}. I compare our framework with the original \texttt{gpt-4o} model and a version of our framework without the assertion-based debugging. The results show that our framework outperforms both baselines, achieving a pass@1 rate of 88.27\% on the HumanEvalPlus benchmark, which is close to 89.6\% reported by the state of the art prompting method QualityFlow\cite{QualityFlow}.

\begin{table}[ht]
    \centering
    \caption{Pass@1 Results on HumanEvalPlus}
    \begin{tabular}{l c c c}
        & pass@1 \\
        \hline
        GPT-4o & 83.43 \\
        Ours (w/o Assert) & 85.18 \\
        Ours (w/ Assert) & 88.27 \\
    \end{tabular}
\end{table}

Comparing to the original \texttt{gpt-4o} model \footnote{Instructed the same as our programmer agent. }, our framework achieves a 4.84\% improvement in pass@1 rate. I also conducted an ablation study to evaluate the impact of assertion-based debugging on the performance of our framework. The results show that the pass@1 rate of our framework without assertion-based debugging is 85.18\%, which is 3.09\% loIr than the version with assertion-based debugging. This indicates that the assertion-based debugging method is effective in improving the correctness of LLM-generated code.

Our implementation and testing results are available at Github\footnote{\href{https://github.com/YuantianDing/AssertDBG.git}{https://github.com/YuantianDing/AssertDBG.git}}. Due to the limited time and resources, I only test our framework on the HumanEvalPlus benchmark. I plan to extend our framework to other benchmarks such as MBPPPlus\cite{evalplus} and MHPP\cite{MHPP} in the future. 

\section{Contribution}

In this project, I have:

\begin{itemize}
    \item Proposed a novel assertion-based debugging framework for LLM-generated code, which enhances the reliability and correctness of synthesized code.
    \item Designed a multi-agent architecture that incorporates assertion-guided strategies, enabling efficient bug detection and localization during runtime.
    \item Implemented a multi-agent system that includes a divide-and-conquer agent for problem decomposition, programmer agents for code generation with assertions, and a debugger agent for iterative analysis of assertion failures.
    \item Conducted experiments on the HumanEvalPlus benchmark to evaluate the effectiveness of our framework.
\end{itemize}

\section{Conclusion and Future Directions}


In this project, I proposed an assertion-based debugging framework that enhances the reliability and correctness of LLM-generated code. By leveraging a multi-agent architecture, I demonstrated how assertion-guided strategies can improve bug detection and localization during runtime. Our experimental results on the HumanEvalPlus benchmark showed a significant performance gain, highlighting the effectiveness of our approach.

However, our project is not complete because of the limited time and resources. The experimental results are only conducted on the HumanEvalPlus benchmark, and even though it has been shown to have a lot randomness, I don't have the resourcaes to test it repeatedly. In the future, I plan to extend our framework to other benchmarks such as MBPPPlus\cite{evalplus} and MHPP\cite{MHPP}. I also aim to explore the integration of assertion-based debugging with other self-debugging and multi-agent collaboration techniques to further enhance the robustness and efficiency of program synthesis.

\section{LLM Usage Acknowledgements}

I have used ChatGPT and Github Copilot to complete this term paper. Here are the links to my conversations. I also put my usage of Github Copilot in the same place:

\begin{itemize}
    \item \href{https://chatgpt.com/share/67fc7d4f-f974-8006-9237-2e2d4fe3eadb}{https://chatgpt.com/share/67fc7d4f-f974-8006-9237-2e2d4fe3eadb}
    \item \href{https://chatgpt.com/share/67fc7cd5-d450-8006-af1c-df824848c97f}{https://chatgpt.com/share/67fc7cd5-d450-8006-af1c-df824848c97f}
    \item \href{https://chatgpt.com/share/67fc7d5d-ea08-8006-a7fc-adf3aa387d13}{https://chatgpt.com/share/67fc7d5d-ea08-8006-a7fc-adf3aa387d13}
    \item \href{https://chatgpt.com/share/67cb68ef-9168-8006-8ad2-7fe15217ad08}{https://chatgpt.com/share/67cb68ef-9168-8006-8ad2-7fe15217ad08}
    \item \href{https://chatgpt.com/share/67cb6920-14c8-8006-937e-e3a9c38144a7}{https://chatgpt.com/share/67cb6920-14c8-8006-937e-e3a9c38144a7}
\end{itemize}

However, I failed to track some of the text generated by Github Copilot, which I borrowed from the Checkpoint 2 writeup. It is mainly in Section 1 \& 2.




% \section*{Accessibility}
% Authors are kindly asked to make their submissions as accessible as possible for everyone including people with disabilities and sensory or neurological differences.
% Tips of how to achieve this and what to pay attention to will be provided on the conference website \url{http://icml.cc/}.

% \section*{Software and Data}

% If a paper is accepted, we strongly encourage the publication of software and data with the
% camera-ready version of the paper whenever appropriate. This can be
% done by including a URL in the camera-ready copy. However, \textbf{do not}
% include URLs that reveal your institution or identity in your
% submission for review. Instead, provide an anonymous URL or upload
% the material as ``Supplementary Material'' into the OpenReview reviewing
% system. Note that reviewers are not required to look at this material
% when writing their review.



% \section*{Acknowledgements}

% \textbf{Do not} include acknowledgements in the initial version of
% the paper submitted for blind review.

% If a paper is accepted, the final camera-ready version can (and
% usually should) include acknowledgements.  Such acknowledgements
% should be placed at the end of the section, in an unnumbered section
% that does not count towards the paper page limit. Typically, this will 
% include thanks to reviewers who gave useful comments, to colleagues 
% who contributed to the ideas, and to funding agencies and corporate 
% sponsors that provided financial support.

% \section*{Impact Statement}

% Authors are \textbf{required} to include a statement of the potential 
% broader impact of their work, including its ethical aspects and future 
% societal consequences. This statement should be in an unnumbered 
% section at the end of the paper (co-located with Acknowledgements -- 
% the two may appear in either order, but both must be before References), 
% and does not count toward the paper page limit. In many cases, where 
% the ethical impacts and expected societal implications are those that 
% are well established when advancing the field of Machine Learning, 
% substantial discussion is not required, and a simple statement such 
% as the following will suffice:

% ``This paper presents work whose goal is to advance the field of 
% Machine Learning. There are many potential societal consequences 
% of our work, none which we feel must be specifically highlighted here.''

% The above statement can be used verbatim in such cases, but we 
% encourage authors to think about whether there is content which does 
% warrant further discussion, as this statement will be apparent if the 
% paper is later flagged for ethics review.

% \nocite{langley00}

\bibliography{example_paper}
\bibliographystyle{icml2025}

\include{appendix}

\end{document}