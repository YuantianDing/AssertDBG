
\subsection{Related Work}


Program synthesis and code generation is certainly a wide-researched field, with a plethora of approaches and methodologies. In this section, I have selected several notable work both from top conferences and from state-of-the-art leaderboards \cite{PaperWithCode}.

\textsc{Self-Debugging} \cite{TeachSelfDebug} introduces an innovative approach to enhancing the accuracy of code generated by large language models (LLMs). The core idea involves enabling LLMs to autonomously execute their generated code using external programs, analyze the outcomes, identify errors, and subsequently refine their code based on this self-assessment. This self-debugging process allows the model to iteratively improve its code without external feedback, such as unit tests or human instructions. However, the self-debugging approach may failed to utilizing runtime execution information to enhance the accuracy and reliability of code generation, as mentioned by other works \cite{LDB}.

Large Language Model Debugger (LDB) \cite{LDB} is a framework designed to emulate human debugging practices by integrating runtime execution information into the debugging process. Unlike traditional methods that treat programs as monolithic units, LDB divides programs into smaller segments, allowing for more granular analysis and refinement. By adopting this step-by-step approach, LDB effectively mirrors human debugging strategies, focusing on smaller code units and utilizing runtime information to enhance the accuracy and reliability of code generation. However, LDB's approach is not generally applicable. It relies heavily on the availability of well-structured intermediate steps and predefined task decomposition, which limits its effectiveness on complex or poorly specified problems. Additionally, LDB does not incorporate assertion-based reasoning or active error localization, making it less effective in cases where silent logic errors occur or when test cases alone fail to capture edge-case behavior. To overcome these limitations, I propose a more flexible and robust framework that integrates assertion-based debugging with multi-agent collaboration, enabling dynamic task decomposition and more accurate error detection through execution-aware reasoning.

Another notable work is MetaGPT \cite{MetaGPT} which leverages large language models (LLMs) to facilitate multi-agent collaboration in complex software engineering tasks. By integrating standardized human workflows into LLM-based multi-agent systems, MetaGPT assigns specialized roles—such as product manager, architect, and engineer—to different agents, mirroring traditional software development pipelines. This role specialization enables the decomposition of intricate tasks into manageable subtasks, promoting coherent and efficient problem-solving. The framework employs an assembly line paradigm, where agents with domain-specific expertise collaborate systematically, reducing errors and enhancing the overall quality of the generated solutions. However, MetaGPT does not explicitly address the issue of runtime error detection and prevention, focusing primarily on role specialization and structured communication. In contrast, our assertion-based debugging framework aims to proactively prevent errors during the synthesis process by incorporating assertion-guided strategies, complementing existing multi-agent methodologies.

QualityFlow \cite{QualityFlow} is a recent state-of-the-art work on \cite{PaperWithCode} that emphasizes the importance of quality assurance in code generation. By introducing a quality control mechanism that evaluates the correctness and robustness of generated code, QualityFlow ensures that the synthesized solutions meet predefined quality standards. The framework employs a feedback loop that iteratively refines the generation process based on quality metrics, enhancing the reliability and accuracy of the final output. While QualityFlow focuses on post-generation quality assessment, our assertion-based debugging framework aims to proactively prevent errors during the synthesis process by incorporating assertion-guided strategies. By integrating runtime checks and assertions into the code generation pipeline, I aim to enhance the reliability and correctness of synthesized code, complementing existing quality assurance methodologies.
