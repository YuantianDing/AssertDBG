
\section{Motivation}

\subsection{Problem Definition}

We follow the problem definition of code generation in \cite{TeachSelfDebug}. Basically, each sample
can be represented as a triplet $(Q, T_v, T_h)$ where $Q$ is the natural language description of the task, $T_v$ is the target code, and $T_h$ is the code generated by the model. $Q$ may include a description of the task, the signature of the function to be implemented, and the constraints that the code must satisfy. In standard code generation tasks, both $Q$ and $T_v$ are provided as inputs, and the objective is to generate a program $P$ that meets the specifications of both $T_v$ and $T_h$. The generated program $P$ is evaluated against both $T_v$ and $T_h$ to check its correctness.



